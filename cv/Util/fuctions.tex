% =============================================================================
% COMANDOS PARA ELEMENTOS DEL CV
% =============================================================================

% Nombre del candidato 
\newcommand{\cvname}[1]{
  \begin{center}
    \fontsize{28pt}{32pt}\selectfont\textcolor{primary}{\textbf{#1}}
  \end{center}
  \vspace{-0.2cm}
}

% Título profesional
\newcommand{\cvtitle}[1]{
  \begin{center}
    \fontsize{14pt}{16pt}\selectfont\textcolor{secondary}{#1}
  \end{center}
  \vspace{0.1cm}
}

% Información de contacto
\newcommand{\cvcontact}[6]{
  \begin{center}
    \begin{tabular}{@{}c@{\hspace{0.5em}}l@{\hspace{1.5em}}c@{\hspace{0.5em}}l@{\hspace{1.5em}}c@{\hspace{0.5em}}l@{}}
      \textcolor{accent}{\faEnvelope} & \href{mailto:#1}{\textcolor{darkgray}{#1}} &
      \textcolor{accent}{\faGlobe} & \href{#6}{\textcolor{darkgray}{Web Page}} &
      \textcolor{accent}{\faMapMarker} & \textcolor{darkgray}{#3} \\[0.3em]
      \textcolor{accent}{\faPhone} & \textcolor{darkgray}{#2} &
      \textcolor{accent}{\faLinkedin} & \href{#4}{\textcolor{darkgray}{julara}} &
      \textcolor{accent}{\faGithub} & \href{https://github.com/#5}{\textcolor{darkgray}{#5}} \\
    \end{tabular}
  \end{center}
  \vspace{0.3cm}
}

% Secciones principales del CV 
\newcommand{\cvsection}[1]{
  \vspace{0.4cm}
  \begin{tikzpicture}
    \draw[fill=primary] (0,0) rectangle (0.15,0.6);
    \node[anchor=west, xshift=0.3cm] at (0.2,0.3) {\textcolor{primary}{\Large\textbf{#1}}};
    \draw[primary, line width=1.2pt] (0.65,-0.1) -- (\textwidth,-0.1); 
  \end{tikzpicture}
  \vspace{0.1cm}
}

% Subsecciones del CV
\newcommand{\cvsubsection}[1]{
  \vspace{0.2cm}
  \textcolor{darkgray}{\large\textbf{#1}}
  \vspace{0.1cm}
}

% Carta de educación
\newcommand{\cveducation}[9]{
  \begin{tabularx}{\textwidth}{X r}
    {\qquad \large{\textbf{#1}}} & \textcolor{secondary}{\textbf{#2}} \\
    {\qquad \textit{#3}}, #4 & \\
    \qquad{Emphasis on #5} & GPA: #6 \\
    \qquad Director: \href{mailto:#8}{#9} & \textcolor{primary}{Detailed List of Exams}
  \end{tabularx}
  \vspace{0.4cm}
}

% Carta de experiencia
\newcommand{\cvexperience}[7]{
  \begin{tabularx}{\textwidth}{X r}
    {\qquad \large{\textbf{#1}}} & \textcolor{secondary}{\textbf{#2}} \\
    {\qquad \large{\textit{#3}}} & \textit{#4} \\
  \end{tabularx}
  \vspace{0.2em}
  \begin{adjustwidth}{2.5em}{2.5em}
    
    #5
    
    \vspace{0.5em}
    \textbf{Supervisor:} \href{mailto:#6}{#7}
  \end{adjustwidth}
  \vspace{0.5cm}
}

% Carta de cursos
\newcommand{\cvtraining}[7]{
  \begin{tabularx}{\textwidth}{X r}
    {\qquad {\textbf{#1}}} & \textcolor{secondary}{\textbf{#2}} \\
    {\qquad {\textit{#3}}} & \textit{#4} \\
  \end{tabularx}
  \vspace{0.1em}
  \begin{adjustwidth}{2.5em}{2.5em}
    
    #5
    
    \vspace{0.5em}
    {\small \textbf{Credential ID:} \href{#6}{\textcolor{primary}{#7}}}
  \end{adjustwidth}
  \vspace{0.6cm}
}

% Etiqueta individual para habilidades
\newcommand{\cvtag}[2][]{
  \tikz[baseline]{
    \node[anchor=base, draw=secondary!70, fill=secondary!10, rounded corners=3pt, 
          inner xsep=6pt, inner ysep=1pt, text height=1.5ex, text depth=.25ex] {#2};
    \ifx&#1&%
    \else
      % Add tooltip functionality if provided
      \node[anchor=base, text width=3cm, align=center, rounded corners=2pt,
            inner sep=2pt, draw=accent!50, fill=white!90, font=\tiny\sffamily,
            opacity=0.9] at (0,0.5) {#1};
    \fi
  } \hspace{1pt}
}

% Mantener compatibilidad con el comando anterior
\renewcommand{\cvtag}[1]{\cvtaglevel[0]{#1}}

% Enhanced project command with date, role, technologies, and description
\newcommand{\cvproject}[5]{
    \vspace{0.1cm}
  \begin{tabularx}{\textwidth}{X r}
    {\qquad \large{\textbf{#1}}} & \textcolor{secondary}{\textbf{#2}} \\
    {\qquad {\textit{#3}}} & \\
  \end{tabularx}
  \vspace{0.1em}
  \begin{adjustwidth}{2.5em}{2.5em}
    \textbf{\textcolor{accent}{Technologies:}} #4
    
    \vspace{0.5em}
    #5
  \end{adjustwidth}
  \vspace{0.6cm}
}

% Habilidad con barra de progreso y nivel textual
\newcommand{\cvskill}[3]{
  \textbf{#1} & 
  \begin{tikzpicture}[baseline]
    \fill[lightgray, rounded corners=2pt] (0,0) rectangle (5,0.25);
    \fill[primary, rounded corners=2pt] (0,0) rectangle (#2*5,0.25);
  \end{tikzpicture} & 
  \textit{\small \textcolor{darkgray}{#3}} \\[0.3cm]
}

% Key technical skill with icon and brief description
\newcommand{\cvkeytechskill}[3]{
  \begin{minipage}{0.48\textwidth}
    \begin{tabularx}{\linewidth}{l X}
      \textcolor{primary}{#1} & \textbf{#2} \\
      & \textit{\small #3} \\
    \end{tabularx}
  \end{minipage}
  \hfill
}

% Create category function with consistent styling
\newcommand{\cvtechcategory}[2]{
  \begin{adjustwidth}{2.5em}{0em}
    {\large\textbf{\textcolor{primary}{#1}}}
    \begin{adjustwidth}{0em}{0em}
      \textit{#2}
    \end{adjustwidth}
    \vspace{0.5cm}
  \end{adjustwidth}
}

% Modified version of cvtaggroup with better spacing and alignment
\newcommand{\cvskilltags}[2]{
  \begin{adjustwidth}{2.5em}{3em}
    \textbf{\textcolor{secondary}{#1:}} \\[0.2cm]
    #2 \\[0.3cm]
  \end{adjustwidth}
}

\newcommand{\cvskillsection}[1]{
  \begin{adjustwidth}{2.5em}{0em}
    \textbf{#1} & & \\[0.1cm]
  \end{adjustwidth}
}

% Grupo de etiquetas con título de categoría
\newcommand{\cvtaggroup}[2]{
  \textbf{\textcolor{secondary}{#1:}} \\[0.1cm]
  #2 \\[0.3cm]
}

% Award entry with organization, date, and description
\newcommand{\cvaward}[4]{
  \begin{tabularx}{\textwidth}{X r}
    {\qquad \large{\textbf{#1}}} & \textcolor{secondary}{\textbf{#3}} \\
    {\qquad \textit{#2}} & \\
  \end{tabularx}
  \vspace{0.1em}
  \begin{adjustwidth}{2.5em}{2.5em}
    #4
  \end{adjustwidth}
  \vspace{0.4cm}
}

% Tag con nivel de intensidad (0=normal, 3=máxima importancia)
\newcommand{\cvtaglevel}[2][0]{
  \tikz[baseline]{
    \ifcase#1
      % Nivel 0 (predeterminado)
      \node[anchor=base, draw=tag-bg-0!70, fill=tag-bg-0!30, rounded corners=3pt, 
          inner xsep=6pt, inner ysep=2pt, text height=1.5ex, text depth=.25ex, 
          text=darkgray] {#2};
    \or
      % Nivel 1
      \node[anchor=base, draw=tag-bg-1!80, fill=tag-bg-1!70, rounded corners=3pt, 
          inner xsep=6pt, inner ysep=2pt, text height=1.5ex, text depth=.25ex, 
          text=primary] {#2};
    \or
      % Nivel 2
      \node[anchor=base, draw=tag-bg-2!90, fill=tag-bg-2!80, rounded corners=3pt, 
          inner xsep=6pt, inner ysep=2pt, text height=1.5ex, text depth=.25ex, 
          text=primary, font=\bfseries] {#2};
    \or
      % Nivel 3 (máximo)
      \node[anchor=base, draw=secondary!90, fill=secondary!90, rounded corners=3pt, 
          inner xsep=6pt, inner ysep=2pt, text height=1.5ex, text depth=.25ex, 
          text=white, font=\bfseries] {#2};
    \fi
  } \hspace{3pt}
}

\newcommand{\cvresearchinterest}[2]{
  \vspace{0.1cm}
  \textcolor{primary}{\faSearch} \textbf{#1}
  \begin{adjustwidth}{1.5em}{0em}
    \textit{\small #2}
  \end{adjustwidth}
  \vspace{0.3cm}
}